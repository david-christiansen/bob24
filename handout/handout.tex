\documentclass[11pt,DIV=16,parskip=half,headings=normal,abstract]{scrartcl}

\usepackage[T3,T1]{fontenc}
\usepackage[utf8]{inputenc}
\usepackage[english]{babel}
\usepackage{amsmath}
\usepackage{amsfonts}
\usepackage[standard]{ntheorem}
\usepackage{amscd}
\usepackage{mathtools}
\usepackage{stmaryrd}
\usepackage{faktor}
\usepackage{enumerate}
\usepackage{wrapfig}
\usepackage[safe]{tipa} % for \textlambda
\usepackage{adjustbox}
\usepackage[para]{footmisc}
\usepackage{paralist}

\usepackage[
	pdfauthor={David Thrane Christiansen, Joachim Breitner},
	pdftitle={Lean for the Functional Programmer -- handout},
	pdfsubject={}
	]{hyperref}
\usepackage{listings}
\usepackage{multicol}
\usepackage{ragged2e}
%\usepackage[para]{footmisc}
\usepackage{fourier}
\usepackage{microtype}
\usepackage{comment}

\usepackage{tikz}
\usetikzlibrary{shapes.geometric}
\usetikzlibrary{calc}
\usetikzlibrary{shapes.multipart}
\usetikzlibrary{arrows}

\urlstyle{sf}

% \author{Joachim Breitner\footnote{\href{mailto:breitner@kit.edu}{breitner@kit.edu}, \url{http://www.joachim-breitner.de/}}}
% \title{Verifikation mit Isabelle}
% \subtitle{(Handout)}
% \publishers{BobKonf 2016}
% \date{19. Februar 2016}

\usepackage{color}
\definecolor{keywordcolor}{rgb}{0.7, 0.1, 0.1}   % red
\definecolor{tacticcolor}{rgb}{0.0, 0.1, 0.6}    % blue
\definecolor{commentcolor}{rgb}{0.4, 0.4, 0.4}   % grey
\definecolor{symbolcolor}{rgb}{0.0, 0.1, 0.6}    % blue
\definecolor{sortcolor}{rgb}{0.1, 0.5, 0.1}      % green
\definecolor{attributecolor}{rgb}{0.7, 0.1, 0.1} % red

\def\lstlanguagefiles{lstlean.tex}
% set default language
\lstset{language=lean}


\definecolor{light-gray}{gray}{0.95}
\lstdefinestyle{basic}{
        ,columns=flexible
        ,basewidth={.365em}
        ,keepspaces=True
	,belowskip=0pt
	,backgroundcolor=\color{light-gray}
	,frame=single
        ,basicstyle=\small\sffamily
        ,commentstyle=\small\sffamily
        ,stringstyle=\itshape
	,rangeprefix=---
	,rangesuffix=---
	,includerangemarker=false
}
\newcommand{\li}{\lstinline}


\pagestyle{plain}
\begin{document}

%\maketitle
\thispagestyle{empty}

\section*{Lean syntax cheat sheet}

\setlength\columnsep{5mm}
\begin{multicols*}{2}
\raggedright
\raggedcolumns

\textbf{Types}

Basic types: \li{Nat}, \li{Bool}, \li{Unit}.\\
Function types: \li{a → b} (typed as \li{\to }).\\
Container types: \li{Option a}, \li{List a}.

\textbf{Definitions}

Simple definitions:
\begin{lstlisting}
def answer : Nat := 42
def double (n : Nat) : Nat := 2 * n
\end{lstlisting}

Function application by juxtaposition:
\begin{lstlisting}
def quadruple (n : Nat) : Nat :=
  double (double n)
\end{lstlisting}

Conditionals and pattern matching:
\begin{lstlisting}
def isAnswer (n : Nat) : Bool :=
  if n = 42 then true else false

def isEmpty (xs : List α) : Bool :=
  match xs with
  | []     => true
  | _ :: _ => false

def isEmpty' : List α → Bool
  | []     => true
  | _ :: _ => false
\end{lstlisting}

Recursion:
\begin{lstlisting}
def length : List α → Nat
  | []      => 0
  | _ :: xs => 1 + length xs
\end{lstlisting}

\textbf{Inductive data types}

\begin{lstlisting}
inductive Tree α where
  | leaf : Tree α
  | branch : Tree α → α → Tree α → Tree α
\end{lstlisting}

\textbf{Useful commands}
\begin{lstlisting}
#eval double 5
#check Tree.branch
\end{lstlisting}

\textbf{Theorems and proofs}

Equality:
\begin{lstlisting}
theorem answerIs42 : answer = 42 := by
  refl
\end{lstlisting}

\goodbreak

Inductive propositions:
\begin{lstlisting}
inductive Rpts (x : α) : List α → Prop where
  | nil : Rpts x []
  | cons : Rpts x xs → Rpts x (x :: xs)
\end{lstlisting}

Proof by induction, unfolding function definitions and application of constructors:
\begin{lstlisting}
theorem Rpts_map (x : α) (xs : List α)
    (f : α → β) (h : Rpts x xs) :
    Rpts (f x) (xs.map f) := by
  induction h
  case nil => exact Rpts.nil
  case cons xs h IH =>
    -- IH : Rpts (f x) (List.map f xs)
    -- ⊢ Rpts (f x) (List.map f (x :: xs))
    unfold List.map
    -- ⊢ Rpts (f x) (f x :: List.map f xs)
    apply Rpts.cons
    -- ⊢ Rpts (f x) (List.map f xs)
    assumption -- uses IH
\end{lstlisting}


\textbf{Further reading and help}
\begin{itemize}
\item Functional Programming in Lean %\url{https://lean-lang.org/functional_programming_in_lean/}
\item Loogle: \href{https://loogle.lean-lang.org/}{loogle.lean-lang.org}
\item Zulip chat: \href{https://leanprover.zulipchat.com/}{leanprover.zulipchat.com}
\item Mathlib4 on Github %: \href{https://github.com/leanprover-community/mathlib4}
\end{itemize}



\end{multicols*}

\end{document}
